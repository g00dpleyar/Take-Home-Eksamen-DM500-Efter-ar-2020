\documentclass[a4paper,12pt]{article}
\usepackage[utf8]{inputenc}
\usepackage{graphicx}
\graphicspath{ {images/} }
\usepackage{hyperref}
\usepackage{mathtools}
\usepackage{amsfonts}

\title{\LaTeX{}}
\author{Andreas Twisttmann Askholm}
\date{\today}

%comments are preceded by a % sign
%bold text \textbf{}
%italics \textit{}
%underlined \underline{}
%fremhæv \emph{}
%billede \includegraphics{}
%matematik \(...\), $...$ eller \begin{math}...\end{math}

\begin{document}

\maketitle

\section{Opgave 2}
\renewcommand{\labelenumi}{\alph{enumi}}
\renewcommand{\labelenumii}{\arabic{enumii}}
\begin{enumerate}
	\item) Hvilke af følgende udsagn er sande ?
\begin{enumerate}
	\item. \( \forall x \in  \mathbb{N}: \exists y \in \mathbb{N}: x < y\)\\

	Udsagnet er sandt,\\
	der altid kan findes et y der er større end x.\\
	
	\item. \(\forall x \in \mathbb{N}:\exists! y \in \mathbb{N}: x < y\)\\
	
	Udsagnet er ikke sandt,\\
	da der kan findes mere end et y der er større end x.\\
	
	\item. \( \exists y \in \mathbb{N}: \forall x \in \mathbb{N}: x < y\)\\
	
	Udsagnet er ikke sandt,\\
	da der ikke findes et y som er større end alle x.\\
	
\end{enumerate}
	\item) Angiv negeringen af udsagn 1. fra spørgsmål a).\\
	Negerings-operatoren($\neg$) må ikke indgå i dit udsagn.\\
	
	\(\exists x \in \mathbb{N}: \forall y \in \mathbb{N}: x \geq y\)\\
	ved negering af et udtryk, ændres operatorerne til de modsat betydende.
	
\end{enumerate}

\section{opgave 3 matricer}
\begin{enumerate}
	\item) \(\begin{bmatrix}
			1 & 0 & 0 & 0\\
			1 & 1 & 0 & 0\\
			1 & 0 & 1 & 0\\
			1 & 0 & 0 & 1
			\end{bmatrix}\)
			
	\item) \(\begin{bmatrix}
			0 & 1 & 0 & 0\\
			0 & 0 & 1 & 1\\
			0 & 0 & 0 & 0\\
			0 & 1 & 0 & 0
			\end{bmatrix}\)
			
	\item) \(\begin{bmatrix}
			1 & 0 & 1 & 0\\
			0 & 1 & 0 & 1\\
			1 & 0 & 1 & 0\\
			0 & 1 & 0 & 1
			\end{bmatrix}\)
\end{enumerate}


\end{document}