\documentclass[a4paper,12pt]{article}
\usepackage[utf8]{inputenc}
\usepackage{graphicx}
\graphicspath{ {images/} }
\usepackage{hyperref}
\usepackage{mathtools}
\usepackage{amsmath}
\usepackage{amsfonts}
\parindent 0ex



\title{Take Home Eksamen i \LaTeX{} og git}
\author{Andreas Twisttmann Askholm (aaskh20), \\Mikkel Lykke Bentsen (MIBEN20), \\Hanno Hagge (hhagg20)}
\date{\today}

%comments are preceded by a % sign
%bold text \textbf{}
%italics \textit{}
%underlined \underline{}
%fremhæv \emph{}
%billede \includegraphics{}
%matematik \(...\), $...$ eller \begin{math}...\end{math}

\begin{document}

\maketitle
\renewcommand{\labelenumi}{\alph{enumi}}
\renewcommand{\labelenumii}{\arabic{enumii}}
\section{Reeksamen februar 2015}
\subsection{Opgave 1.}
I det følgende lader vi $U = \lbrace1, 2, 3, . . . , 15\rbrace$ være universet (universal set).\\
Betragt de to mængder\\
$$A = {2n \mid n \in S}$$
$$B = {3n + 2 \mid n \in S}$$
hvor S = $\lbrace1, 2, 3, 4\rbrace.$ \\
Angiv samtlige elementer i hver af følgende mængder.
\\\\
\begin{enumerate}


\item) A  \hspace{5mm} Mængden A er alle værdier i  S ganget med 2 (2n).$$ A = \lbrace 2, 4, 6, 8\rbrace$$
\item) B \hspace{5mm} Mængden B er alle værdier i S ganget med 3, og derefter adderet med 2 (3n + 2). $$ B = \lbrace 5, 8, 11, 14 \rbrace$$
\item) $A \cap B$ \hspace{5mm} Fællesmængden af A og B er den mængde bestående af de elementer de har tilfælles. $$ A \cap B = \lbrace8\rbrace$$
\item) $A \cup B$ \hspace{5mm} Foreningsmængden af A og B er mængden bestående af alle elementer fra A og B. Det samme element kan ikke optræde flere gange. $$ A \cup B = \lbrace 2, 4, 5, 6, 8, 11, 14\rbrace$$
\item) $A - B$ \hspace{5mm} Mængden A - B er den mængden A uden de elementer A har tilfælles med B. $$ A - B = \lbrace 2, 4, 6 \rbrace $$
\item) $ \bar{A}$ \hspace{5mm} Komplimentet af A er bestående af alle de elementer i universet som \emph{ikke} er i A. $$ \bar{A} = \lbrace 1, 3, 5, 7, 9, 10, 11, 12, 13, 14, 15 \rbrace$$
\end{enumerate}
\subsection{Opgave 2}

\begin{enumerate}
	\item) Hvilke af følgende udsagn er sande ?
\begin{enumerate}
	\item. \( \forall x \in  \mathbb{N}: \exists y \in \mathbb{N}: x < y\)\\

	Udsagnet er sandt,\\
	der altid kan findes et y der er større end x.\\
	
	\item. \(\forall x \in \mathbb{N}:\exists! y \in \mathbb{N}: x < y\)\\
	
	Udsagnet er ikke sandt,\\
	da der kan findes mere end et y der er større end x.\\
	
	\item. \( \exists y \in \mathbb{N}: \forall x \in \mathbb{N}: x < y\)\\
	
	Udsagnet er ikke sandt,\\
	da der ikke findes et y som er større end alle x.\\
	
\end{enumerate}
	\item) Angiv negeringen af udsagn 1. fra spørgsmål a).\\
	Negerings-operatoren($\neg$) må ikke indgå i dit udsagn.\\
	
	\(\exists x \in \mathbb{N}: \forall y \in \mathbb{N}: x \geq y\)\\
	ved negering af et udtryk, ændres operatorerne til de modsat betydende.
	
\end{enumerate}

\subsection{Opgave 3}
Lad R, S og T være binære relationer på mængden $\lbrace1,2,3,4 \rbrace$

\begin{enumerate}

	\item) Lad\textit{R} = $\lbrace(1,1),(2,1),(2,2),(2,4),(3,1),(3,3), (3,4),(4,1),(4,4)\rbrace$.
	\\Er \textit{R} en partiel ordning?
	\\
	\\For at en relation er en partiel ordning, skal den være både refleksiv, transitiv og asymetrisk. \textit{R} er refleksiv, da ethvert element er relateret til sig selv:$\lbrace(1,1),(2,2),(3,3),(4,4)\rbrace$. Den er asymetrisk, fordi ingen af relationerne har der har (a,b) også har (b,a), untagen hvis de er transitiv. Relationen er transitiv, da a er relatieret til b og b er relateret til c, er a også relateret til c.\\
	
	
	
	\item) Lad \textit{S} = $\lbrace(1, 2),(2, 3),(2, 4),(4, 2)\rbrace.$
Angiv den transitive lukning af \textit{S}.
\\For at lukke relationen transitiv, skal hvis man har (a,b) og (b,c) også have (a,c). I vores relation ville det være: 
\\(1,3), (2,2), (1,4).
\\
\\
	\item) Lad \textit{T} = $\lbrace(1, 1),(1, 3),(2, 2),(2, 4),(3, 1),(3, 3),(4, 2),(4, 4)\rbrace$.
\\Bemærk, at \textit{T} er en ækvivalens-relation.
\\Angiv \textit{T}'s ækvivalens-klasser.
\\
\\Ækvivalensklasser er relationer inden for en ækvivalensrelation, der ikke har direkt forbindelse til resten. I dette eksempel er
\\ækvivalensklasserne:
\\ 1 og 3 
\\ 2 og 4

\end{enumerate}

\subsubsection{opgave 3 matricer}
\begin{enumerate}
	\item) \(\begin{bmatrix}
			1 & 0 & 0 & 0\\
			1 & 1 & 0 & 0\\
			1 & 0 & 1 & 0\\
			1 & 0 & 0 & 1
			\end{bmatrix}\)
			
	\item) \(\begin{bmatrix}
			0 & 1 & 0 & 0\\
			0 & 0 & 1 & 1\\
			0 & 0 & 0 & 0\\
			0 & 1 & 0 & 0
			\end{bmatrix}\)
			
	\item) \(\begin{bmatrix}
			1 & 0 & 1 & 0\\
			0 & 1 & 0 & 1\\
			1 & 0 & 1 & 0\\
			0 & 1 & 0 & 1
			\end{bmatrix}\)
\end{enumerate}


\section{Opgaver fra reeksamen Januar 2012}
\subsection{Opgave 1}
Betragt funktionerne $f : R \longrightarrow$ R og g $: R \longrightarrow R $ defineret ved 
$$f(x) = x^2 + x + 1$$
$$g(x) = 2x - 2$$
\begin{enumerate}

\item) Er f en bijektion?  \\\\
f er ikke en bijektion da den værken er sujektiv eller injektiv. Den dækker ikke alle værdier af y og der er flere x værdier tilknyttet en y værdi.
\\
\\
\item) Har f en invers funktion?
\\
\\
Den inverse funktion for f er ikke defineret da en funktion ikke kan have flere værdier af y for en x værdi.
\\
\\
\item) Angiv f + g.
$$(x^2 + x + 1) + (2x-2)$$
\item) Angiv g ◦ f.
$$2(x^2+x+1) - 2$$
\end{enumerate}

\end{document}